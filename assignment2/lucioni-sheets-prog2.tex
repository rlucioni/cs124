\documentclass[solution, letterpaper]{cs121}

\usepackage{graphicx}

%% Please fill in your name and collaboration statement here.
%\newcommand{\studentName}{Renzo Lucioni and Daniel Broudy}
%\newcommand{\collaborationStatement}{I collaborated with...}
\newcommand{\solncolor}{red}
\begin{document}

\header{2}{March 5, 2013, at 11:40 AM}{}{}

%%%%%%%%%%%%%%%%%%%%%%%%%%%%%%%%%%%%%%%%%%%%%%%%%%%%
\section*{Analytical Solution}

The conventional algorithm requires $n^3$ multiplications and $n^2(n-1)$ additions. Thus, the work required by the conventional algorithm is $n^3 + n^2(n-1)$, when $n \leq n_0$.

Strassens's algorithm requires 7 matrix products per recurrence, 10 additions and subtractions when deriving the $P_1, \ldots, P_7$, and 8 additions and subtractions when manipulating the $P_i$'s to find the matrix sums. Therefore, the work required by Strassen's algorithm is represented by the recurrence $T(n) = 7T(\frac{n}{2}) + (10+8)\frac{n^2}{4} = 7T(\frac{n}{2}) + \frac{9}{2}n^2$ where $T(1) = 1$, when $n > n_0$. Using Mathematica, we find that the solution to this recurrence is $T(n) = 7^{(\log_2 n) + 1}-6 n^2$.

This gives us the following function, which we want to minimize over all $n$:
\[
    f(n)= 
\begin{cases}
    n^3 + n^2(n-1), & \text{if } n \leq n_0\\
    7^{(\log_2 n) + 1}-6 n^2, & \text{if } n > n_0 \\
\end{cases}
\]

We can analytically determine the value of $n_0$ that optimizes the running time of the variant of Strassen's we are studying by setting both cases equal in $f$ equal and solving for $n$. Using Mathematica, we see that both sides are equal when $n = 1$ and $n \approx 654.031$. The former value of $n$ is uninteresting, since we know that the base case of Strassen's is the same as the conventional method. Thus, $n_0 \approx 654.031$.

\section*{Experimental Results}

\section*{Discussion}

\end{document}



